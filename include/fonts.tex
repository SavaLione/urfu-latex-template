%%% Кодировки и шрифты %%%
\ifxetexorluatex
    % Язык по-умолчанию русский с поддержкой приятных команд пакета babel
    \setmainlanguage[babelshorthands=true]{russian}

    % Дополнительный язык = английский (в американской вариации по-умолчанию)
    \setotherlanguage{english}

    % Проверка существования шрифтов. Недоступна в pdflatex
    \ifnumequal{\value{fontfamily}}{1}{
        \IfFontExistsTF{Times New Roman}{}{\setcounter{fontfamily}{0}}
    }{}
    \ifnumequal{\value{fontfamily}}{2}{
        \IfFontExistsTF{LiberationSerif}{}{\setcounter{fontfamily}{0}}
    }{}

    %   setmonofont - моноширинный шрифт
    %   newfontfamily\cyrillicfonttt - моноширинный шрифт для кириллицы
    %   defaultfontfeatures - стандартные лигатуры TeX, замены нескольких дефисов на тире и т. п.
    %                         Настройки моноширинного шрифта должны идти до этой строки, чтобы при врезках кода программ в коде не применялись лигатуры и замены дефисов
    %   setmainfont - Шрифт с засечками
    %   newfontfamily\cyrillicfont - Шрифт с засечками для кириллицы
    %   setsansfont - Шрифт без засечек
    %   newfontfamily\cyrillicfontsf - Шрифт без засечек для кириллицы

    % Семейство шрифтов CMU. Используется как fallback
    \ifnumequal{\value{fontfamily}}{0}{
        \setmonofont{CMU Typewriter Text}
        \newfontfamily\cyrillicfonttt{CMU Typewriter Text}
        \defaultfontfeatures{Ligatures=TeX}
        \setmainfont{CMU Serif}
        \newfontfamily\cyrillicfont{CMU Serif}
        \setsansfont{CMU Sans Serif}
        \newfontfamily\cyrillicfontsf{CMU Sans Serif}
    }

    % Семейство MS шрифтов
    \ifnumequal{\value{fontfamily}}{1}{
        \setmonofont{Courier New}
        \newfontfamily\cyrillicfonttt{Courier New}
        \defaultfontfeatures{Ligatures=TeX}
        \setmainfont{Times New Roman}
        \newfontfamily\cyrillicfont{Times New Roman}
        \setsansfont{Arial}
        \newfontfamily\cyrillicfontsf{Arial}
    }

    % Семейство шрифтов Liberation (https://pagure.io/liberation-fonts)
    \ifnumequal{\value{fontfamily}}{2}{
        \setmonofont{LiberationMono}[Scale=0.87]
        \newfontfamily\cyrillicfonttt{LiberationMono}[Scale=0.87]
        \defaultfontfeatures{Ligatures=TeX}
        \setmainfont{LiberationSerif}
        \newfontfamily\cyrillicfont{LiberationSerif}
        \setsansfont{LiberationSans}
        \newfontfamily\cyrillicfontsf{LiberationSans}
    }

\else
    % Используется pscyr, при наличии
    \ifnumequal{\value{usealtfont}}{1}{
        \IfFileExists{pscyr.sty}{\renewcommand{\rmdefault}{ftm}}{}
    }{}
\fi
