%%%%%%%%%%%%%%%%%%%%%%%%%%%%%%%%%%%%%%%%%%%%%%%%%%%%%%
%%%% Файл упрощённых настроек шаблона диссертации %%%%
%%%%%%%%%%%%%%%%%%%%%%%%%%%%%%%%%%%%%%%%%%%%%%%%%%%%%%

%%% Инициализирование переменных, не трогать!  %%%
\newcounter{intvl}
\newcounter{otstup}
\newcounter{contnumeq}
\newcounter{contnumfig}
\newcounter{contnumtab}
\newcounter{pgnum}
\newcounter{chapstyle}
\newcounter{headingdelim}
\newcounter{headingalign}
\newcounter{headingsize}
%%%%%%%%%%%%%%%%%%%%%%%%%%%%%%%%%%%%%%%%%%%%%%%%%%%%%%

%%% Область упрощённого управления оформлением %%%

%% Интервал между заголовками и между заголовком и текстом %%
% Заголовки отделяют от текста сверху и снизу
% тремя интервалами (ГОСТ Р 7.0.11-2011, 5.3.5)

% Коэффициент кратности к размеру шрифта
\setcounter{intvl}{3}

%% Отступы у заголовков в тексте %%
% 0 --- без отступа
% 1 --- абзацный отступ
\setcounter{otstup}{1}

%% Нумерация формул, таблиц и рисунков %%
% Нумерация формул
% 0 --- пораздельно (во введении подряд, без номера раздела);
% 1 --- сквозная нумерация по всей диссертации
\setcounter{contnumeq}{0}


% Нумерация рисунков
% 0 --- пораздельно (во введении подряд, без номера раздела);
% 1 --- сквозная нумерация по всей диссертации
\setcounter{contnumfig}{1}


% Нумерация таблиц
% 0 --- пораздельно (во введении подряд, без номера раздела);
% 1 --- сквозная нумерация по всей диссертации
\setcounter{contnumtab}{1}

%% Оглавление %%
% 0 --- номера страниц никак не обозначены;
% 1 --- Стр. над номерами страниц (дважды компилировать после изменения настройки) до какого уровня подразделов выносить в оглавление до какого уровня нумеровать подразделы
\setcounter{pgnum}{0}

% До какого уровня подразделов выносить в оглавление
\settocdepth{subsection}

% До какого уровня нумеровать подразделы
\setsecnumdepth{subsection}

%% Текст и форматирование заголовков %%

% 0 --- разделы только под номером;
% 1 --- разделы с названием "Глава" перед номером
\setcounter{chapstyle}{0}

% 1 --- номера разделов и приложений отделены точкой с пробелом, подразделы пропуском без точки;
% 2 --- номера разделов, подразделов и приложений отделены точкой с пробелом.
% 0 --- номер отделен пропуском в 1em или \quad;
\setcounter{headingdelim}{0}

%% Выравнивание заголовков в тексте %%

% 0 --- по центру;
% 1 --- по левому краю
\setcounter{headingalign}{1}


%% Размеры заголовков в тексте %%
% 1 --- пропорционально изменяющийся размер в зависимости от базового шрифта
% 0 --- по ГОСТ, все всегда 14 пт;
\setcounter{headingsize}{0}

%% Подпись таблиц %%

% Смещение строк подписи после первой строки
\newcommand{\tabindent}{0cm}

% Тип форматирования заголовка таблицы:
% plain --- название и текст в одной строке
% split --- название и текст в разных строках
\newcommand{\tabformat}{plain}

%%% Настройки форматирования таблицы `plain`

% Выравнивание по центру подписи, состоящей из одной строки:
% true  --- выравнивать
% false --- не выравнивать
\newcommand{\tabsinglecenter}{false}

% Выравнивание подписи таблиц:
% justified   --- выравнивать как обычный текст («по ширине»)
% centering   --- выравнивать по центру
% centerlast  --- выравнивать по центру только последнюю строку
% centerfirst --- выравнивать по центру только первую строку (не рекомендуется)
% raggedleft  --- выравнивать по правому краю
% raggedright --- выравнивать по левому краю
\newcommand{\tabjust}{justified}

% Разделитель записи «Таблица #» и названия таблицы
\newcommand{\tablabelsep}{~\cyrdash\ }

%%% Настройки форматирования таблицы `split`

% Положение названия таблицы:
% \centering   --- выравнивать по центру
% \raggedleft  --- выравнивать по правому краю
% \raggedright --- выравнивать по левому краю
\newcommand{\splitformatlabel}{\raggedleft}

% Положение текста подписи:
% \centering   --- выравнивать по центру
% \raggedleft  --- выравнивать по правому краю
% \raggedright --- выравнивать по левому краю
\newcommand{\splitformattext}{\raggedright}

%% Подпись рисунков %%
% Разделитель записи «Рисунок #» и названия рисунка
% (ГОСТ 2.105, 4.3.1)
% "--- здесь не работает
\newcommand{\figlabelsep}{~\cyrdash\ }

%%% Цвета гиперссылок %%%
% Latex color definitions: http://latexcolor.com/

% Default
%\definecolor{linkcolor}{rgb}{0.9,0,0}
%\definecolor{citecolor}{rgb}{0,0.6,0}
%\definecolor{urlcolor}{rgb}{0,0,1}

% Black
\definecolor{linkcolor}{rgb}{0,0,0}
\definecolor{citecolor}{rgb}{0,0,0}
\definecolor{urlcolor}{rgb}{0,0,0}
