%%% Шаблон %%%

% решаем проблему превращения
% названия цвета в результате \MakeUppercase,
% http://tex.stackexchange.com/a/187930,
% \DeclareRobustCommand protects \fixme
% from expanding inside \MakeUppercase
\DeclareRobustCommand{\fixme}{\textcolor{red}}

% Абзацный отступ. Должен быть одинаковым по всему тексту и равен пяти знакам (ГОСТ Р 7.0.11-2011, 5.3.7).
\AtBeginDocument{\setlength{\parindent}{2.5em}}

%%% Таблицы %%%

% нумерация таблиц
\DeclareCaptionLabelSeparator{tabsep}{\tablabelsep}

\DeclareCaptionFormat{split}{\splitformatlabel#1\par\splitformattext#3}


% format - формат подписи (plain|hang)
% font - нормальные размер, цвет, стиль шрифта
% skip - отбивка под подписью
% parskip - отбивка между параграфами подписи
% position - положение подписи
% justification - центровка
% indent - смещение строк после первой
% labelsep - разделитель
% singlelinecheck - не выравнивать по центру, если умещается в одну строку
\captionsetup[table]{
    format=\tabformat,
    font=normal,
    skip=.0pt,
    parskip=.0pt,
    position=above,
    justification=\tabjust,
    indent=\tabindent,
    labelsep=tabsep,
    singlelinecheck=\tabsinglecenter,
}

%%% Рисунки %%%

% нумерация рисунков
\DeclareCaptionLabelSeparator{figsep}{\figlabelsep}

% format - формат подписи (plain|hang)
% font - нормальные размер, цвет, стиль шрифта
% skip - отбивка под подписью
% parskip - отбивка между параграфами подписи
% position - положение подписи
% singlelinecheck - выравнивание по центру, если умещается в одну строку
% justification - центровка
% labelsep - разделитель
\captionsetup[figure]{
    format=plain,
    font=normal,
    skip=.0pt,
    parskip=.0pt,
    position=below,
    singlelinecheck=true,
    justification=centerlast,
    labelsep=figsep,
}

%%% Подписи подрисунков %%%
\DeclareCaptionSubType{figure}
% нумерация подрисунков
\renewcommand\thesubfigure{\asbuk{subfigure}}

\ifsynopsis
    \DeclareCaptionFont{norm}{\fontsize{10pt}{11pt}\selectfont}
    \newcommand{\subfigureskip}{2.pt}
\else
    \DeclareCaptionFont{norm}{\fontsize{14pt}{16pt}\selectfont}
    \newcommand{\subfigureskip}{0.pt}
\fi

% labelfont - нормальный размер подписей подрисунков
% textfont - нормальный размер подписей подрисунков
% labelsep - разделитель
% labelformat - одна скобка справа от номера
% justification - центровка
% singlelinecheck - выравнивание по центру, если умещается в одну строку
% skip - отбивка над подписью
% parskip - отбивка между параграфами подписи
% position - положение подписи
\captionsetup[subfloat]{
    labelfont=norm,
    textfont=norm,
    labelsep=space,
    labelformat=brace,
    justification=centering,
    singlelinecheck=true,
    skip=\subfigureskip,
    parskip=.0pt,
    position=below,
}

%%% Настройки ссылок на рисунки, таблицы и др. %%%
% команды \cref...format отвечают за форматирование при помощи команды \cref
% команды \labelcref...format отвечают за форматирование при помощи команды \labelcref

\ifpresentation
\else
    \crefdefaultlabelformat{#2#1#3}

    % Уравнение

    % одиночная ссылка с приставкой
    \crefformat{equation}{(#2#1#3)}

    % одиночная ссылка без приставки
    \labelcrefformat{equation}{(#2#1#3)}

    % диапазон ссылок с приставкой
    \crefrangeformat{equation}{(#3#1#4) \cyrdash~(#5#2#6)}

    % диапазон ссылок без приставки
    \labelcrefrangeformat{equation}{(#3#1#4) \cyrdash~(#5#2#6)}

    % перечисление ссылок с приставкой
    \crefmultiformat{equation}{(#2#1#3)}{ и~(#2#1#3)}{, (#2#1#3)}{ и~(#2#1#3)}

    % перечисление без приставки
    \labelcrefmultiformat{equation}{(#2#1#3)}{ и~(#2#1#3)}{, (#2#1#3)}{ и~(#2#1#3)}

    % Подуравнение

    % одиночная ссылка с приставкой
    \crefformat{subequation}{(#2#1#3)}

    % одиночная ссылка без приставки
    \labelcrefformat{subequation}{(#2#1#3)}

    % диапазон ссылок с приставкой
    \crefrangeformat{subequation}{(#3#1#4) \cyrdash~(#5#2#6)}

    % диапазон ссылок без приставки
    \labelcrefrangeformat{subequation}{(#3#1#4) \cyrdash~(#5#2#6)}

    % перечисление ссылок с приставкой
    \crefmultiformat{subequation}{(#2#1#3)}{ и~(#2#1#3)}{, (#2#1#3)}{ и~(#2#1#3)}

    % перечисление без приставки
    \labelcrefmultiformat{subequation}{(#2#1#3)}{ и~(#2#1#3)}{, (#2#1#3)}{ и~(#2#1#3)}

    % Глава

    % одиночная ссылка с приставкой
    \crefformat{chapter}{#2#1#3}

    % одиночная ссылка без приставки
    \labelcrefformat{chapter}{#2#1#3}

    % диапазон ссылок с приставкой
    \crefrangeformat{chapter}{#3#1#4 \cyrdash~#5#2#6}

    % диапазон ссылок без приставки
    \labelcrefrangeformat{chapter}{#3#1#4 \cyrdash~#5#2#6}

    % перечисление ссылок с приставкой
    \crefmultiformat{chapter}{#2#1#3}{ и~#2#1#3}{, #2#1#3}{ и~#2#1#3}

    % перечисление без приставки
    \labelcrefmultiformat{chapter}{#2#1#3}{ и~#2#1#3}{, #2#1#3}{ и~#2#1#3}

    % Параграф

    % одиночная ссылка с приставкой
    \crefformat{section}{#2#1#3}

    % одиночная ссылка без приставки
    \labelcrefformat{section}{#2#1#3}

    % диапазон ссылок с приставкой
    \crefrangeformat{section}{#3#1#4 \cyrdash~#5#2#6}

    % диапазон ссылок без приставки
    \labelcrefrangeformat{section}{#3#1#4 \cyrdash~#5#2#6}

    % перечисление ссылок с приставкой
    \crefmultiformat{section}{#2#1#3}{ и~#2#1#3}{, #2#1#3}{ и~#2#1#3}

    % перечисление без приставки
    \labelcrefmultiformat{section}{#2#1#3}{ и~#2#1#3}{, #2#1#3}{ и~#2#1#3}

    % Приложение

    % одиночная ссылка с приставкой
    \crefformat{appendix}{#2#1#3}

    % одиночная ссылка без приставки
    \labelcrefformat{appendix}{#2#1#3}

    % диапазон ссылок с приставкой
    \crefrangeformat{appendix}{#3#1#4 \cyrdash~#5#2#6}

    % диапазон ссылок без приставки
    \labelcrefrangeformat{appendix}{#3#1#4 \cyrdash~#5#2#6}

    % перечисление ссылок с приставкой
    \crefmultiformat{appendix}{#2#1#3}{ и~#2#1#3}{, #2#1#3}{ и~#2#1#3}

    % перечисление без приставки
    \labelcrefmultiformat{appendix}{#2#1#3}{ и~#2#1#3}{, #2#1#3}{ и~#2#1#3}

    % Рисунок

    % одиночная ссылка с приставкой
    \crefformat{figure}{#2#1#3}


    % одиночная ссылка без приставки
    \labelcrefformat{figure}{#2#1#3}

    % диапазон ссылок с приставкой
    \crefrangeformat{figure}{#3#1#4 \cyrdash~#5#2#6}

    % диапазон ссылок без приставки
    \labelcrefrangeformat{figure}{#3#1#4 \cyrdash~#5#2#6}

    % перечисление ссылок с приставкой
    \crefmultiformat{figure}{#2#1#3}{ и~#2#1#3}{, #2#1#3}{ и~#2#1#3}

    % перечисление без приставки
    \labelcrefmultiformat{figure}{#2#1#3}{ и~#2#1#3}{, #2#1#3}{ и~#2#1#3}

    % Таблица

    % одиночная ссылка с приставкой
    \crefformat{table}{#2#1#3}

    % одиночная ссылка без приставки
    \labelcrefformat{table}{#2#1#3}

    % диапазон ссылок с приставкой
    \crefrangeformat{table}{#3#1#4 \cyrdash~#5#2#6}

    % диапазон ссылок без приставки
    \labelcrefrangeformat{table}{#3#1#4 \cyrdash~#5#2#6}

    % перечисление ссылок с приставкой
    \crefmultiformat{table}{#2#1#3}{ и~#2#1#3}{, #2#1#3}{ и~#2#1#3}

    % перечисление без приставки
    \labelcrefmultiformat{table}{#2#1#3}{ и~#2#1#3}{, #2#1#3}{ и~#2#1#3}

    % Листинг

    % одиночная ссылка с приставкой
    \crefformat{lstlisting}{#2#1#3}

    % одиночная ссылка без приставки
    \labelcrefformat{lstlisting}{#2#1#3}

    % диапазон ссылок с приставкой
    \crefrangeformat{lstlisting}{#3#1#4 \cyrdash~#5#2#6}

    % диапазон ссылок без приставки
    \labelcrefrangeformat{lstlisting}{#3#1#4 \cyrdash~#5#2#6}

    % перечисление ссылок с приставкой
    \crefmultiformat{lstlisting}{#2#1#3}{ и~#2#1#3}{, #2#1#3}{ и~#2#1#3}

    % перечисление без приставки
    \labelcrefmultiformat{lstlisting}{#2#1#3}{ и~#2#1#3}{, #2#1#3}{ и~#2#1#3}

    % Листинг

    % одиночная ссылка с приставкой
    \crefformat{ListingEnv}{#2#1#3}

    % одиночная ссылка без приставки
    \labelcrefformat{ListingEnv}{#2#1#3}

    % диапазон ссылок с приставкой
    \crefrangeformat{ListingEnv}{#3#1#4 \cyrdash~#5#2#6}

    % диапазон ссылок без приставки
    \labelcrefrangeformat{ListingEnv}{#3#1#4 \cyrdash~#5#2#6}

    % перечисление ссылок с приставкой
    \crefmultiformat{ListingEnv}{#2#1#3}{ и~#2#1#3}{, #2#1#3}{ и~#2#1#3}

    % перечисление без приставки
    \labelcrefmultiformat{ListingEnv}{#2#1#3}{ и~#2#1#3}{, #2#1#3}{ и~#2#1#3}
\fi

%%% Настройки гиперссылок %%%
\ifluatex
    % Unicode encoded PDF strings
    \hypersetup{unicode,}
\fi

% linktocpage - ссылки с номера страницы в оглавлении, списке таблиц и списке рисунков
% linktoc - both the section and page part are links
% pdfpagelabels - set PDF page labels (true|false)
% plainpages - Forces page anchors to be named by the Arabic form  of the page number, rather than the formatted form
% colorlinks - отображаются раскрашенным текстом, а не раскрашенным прямоугольником, вокруг текста
% linkcolor - цвет ссылок типа ref, eqref и подобных
% citecolor - цвет ссылок-цитат
% urlcolor - цвет гиперссылок
% hidelinks - links (removing color and border)
% pdftitle -  Заголовок
% pdfauthor - Автор
% pdfsubject - Тема
% pdfcreator - Создатель, Приложение
% pdfproducer - Производитель, Производитель PDF
% pdfkeywords - Ключевые слова
% pdflang
\hypersetup{
    linktocpage=true,
    %    linktoc=all,
    %    pdfpagelabels=false,
    plainpages=false,
    colorlinks,
    linkcolor={linkcolor},
    citecolor={citecolor},
    urlcolor={urlcolor},
    %    hidelinks,
    pdftitle={\thesisTitle},
    pdfauthor={\thesisAuthor},
    pdfsubject={\thesisTheme},
    % pdfcreator={\thesisCreator},
    pdfproducer={\thesisProducer},
    pdfkeywords={\thesisKeywords},
    pdflang={\thesisLanguage},
}
% Черновик
\ifnumequal{\value{draft}}{1}{
    \hypersetup{
        draft,
    }
}{}

%%% Списки %%%
% Используем короткое тире (endash) для ненумерованных списков (ГОСТ 2.105-95, пункт 4.1.7, требует дефиса, но так лучше смотрится)
\renewcommand{\labelitemi}{\normalfont\bfseries{--}}

% Перечисление строчными буквами латинского алфавита (ГОСТ 2.105-95, 4.1.7)
%\renewcommand{\theenumi}{\alph{enumi}}
%\renewcommand{\labelenumi}{\theenumi)}

% Перечисление строчными буквами русского алфавита (ГОСТ 2.105-95, 4.1.7)
\makeatletter
% Управляем списками/перечислениями через пакет enumitem, а он 'не знает' про asbuk, потому 'учим' его
\AddEnumerateCounter{\asbuk}{\russian@alph}{щ}
\makeatother

%первый уровень нумерации
%\renewcommand{\theenumi}{\asbuk{enumi}}

%первый уровень нумерации
%\renewcommand{\labelenumi}{\theenumi)}

%второй уровень нумерации
\renewcommand{\theenumii}{\asbuk{enumii}}

%второй уровень нумерации
\renewcommand{\labelenumii}{\theenumii)}

%третий уровень нумерации
\renewcommand{\theenumiii}{\arabic{enumiii}}

%третий уровень нумерации
\renewcommand{\labelenumiii}{\theenumiii)}

% nosep - Единый стиль для всех списков (пакет enumitem), без дополнительных интервалов.
% labelindent=\parindent,leftmargin=* - Каждый пункт, подпункт и перечисление записывают с абзацного отступа (ГОСТ 2.105-95, 4.1.8)
\setlist{nosep, labelindent=\parindent,leftmargin=*}

%%%%%%% SAVALIONE % Новые стили перечислений
\newlist{enumeratePaper}{enumerate}{10}
\setlist[enumeratePaper]{label*=\arabic*.,itemindent=4em, leftmargin=0em}

\newlist{itemizePaper}{itemize}{10}
\setlist[itemizePaper]{label*=--,itemindent=3.5em, leftmargin=0em}

%%% Правильная нумерация приложений, рисунков и формул %%%
%% По ГОСТ 2.105, п. 4.3.8 Приложения обозначают заглавными буквами русского алфавита,
%% начиная с А, за исключением букв Ё, З, Й, О, Ч, Ь, Ы, Ъ.
%% Здесь также переделаны все нумерации русскими буквами.
\ifxetexorluatex
    \makeatletter
    \def\russian@Alph#1{\ifcase#1\or
            А\or Б\or В\or Г\or Д\or Е\or Ж\or
            И\or К\or Л\or М\or Н\or
            П\or Р\or С\or Т\or У\or Ф\or Х\or
            Ц\or Ш\or Щ\or Э\or Ю\or Я\else\xpg@ill@value{#1}{russian@Alph}\fi}
    \def\russian@alph#1{\ifcase#1\or
            а\or б\or в\or г\or д\or е\or ж\or
            и\or к\or л\or м\or н\or
            п\or р\or с\or т\or у\or ф\or х\or
            ц\or ш\or щ\or э\or ю\or я\else\xpg@ill@value{#1}{russian@alph}\fi}
    \def\cyr@Alph#1{\ifcase#1\or
            А\or Б\or В\or Г\or Д\or Е\or Ж\or
            И\or К\or Л\or М\or Н\or
            П\or Р\or С\or Т\or У\or Ф\or Х\or
            Ц\or Ш\or Щ\or Э\or Ю\or Я\else\xpg@ill@value{#1}{cyr@Alph}\fi}
    \def\cyr@alph#1{\ifcase#1\or
            а\or б\or в\or г\or д\or е\or ж\or
            и\or к\or л\or м\or н\or
            п\or р\or с\or т\or у\or ф\or х\or
            ц\or ш\or щ\or э\or ю\or я\else\xpg@ill@value{#1}{cyr@alph}\fi}
    \makeatother
\else
    \makeatletter
    \if@uni@ode
        \def\russian@Alph#1{\ifcase#1\or
                А\or Б\or В\or Г\or Д\or Е\or Ж\or
                И\or К\or Л\or М\or Н\or
                П\or Р\or С\or Т\or У\or Ф\or Х\or
                Ц\or Ш\or Щ\or Э\or Ю\or Я\else\@ctrerr\fi}
    \else
        \def\russian@Alph#1{\ifcase#1\or
                \CYRA\or\CYRB\or\CYRV\or\CYRG\or\CYRD\or\CYRE\or\CYRZH\or
                \CYRI\or\CYRK\or\CYRL\or\CYRM\or\CYRN\or
                \CYRP\or\CYRR\or\CYRS\or\CYRT\or\CYRU\or\CYRF\or\CYRH\or
                \CYRC\or\CYRSH\or\CYRSHCH\or\CYREREV\or\CYRYU\or
                \CYRYA\else\@ctrerr\fi}
    \fi
    \if@uni@ode
        \def\russian@alph#1{\ifcase#1\or
                а\or б\or в\or г\or д\or е\or ж\or
                и\or к\or л\or м\or н\or
                п\or р\or с\or т\or у\or ф\or х\or
                ц\or ш\or щ\or э\or ю\or я\else\@ctrerr\fi}
    \else
        \def\russian@alph#1{\ifcase#1\or
                \cyra\or\cyrb\or\cyrv\or\cyrg\or\cyrd\or\cyre\or\cyrzh\or
                \cyri\or\cyrk\or\cyrl\or\cyrm\or\cyrn\or
                \cyrp\or\cyrr\or\cyrs\or\cyrt\or\cyru\or\cyrf\or\cyrh\or
                \cyrc\or\cyrsh\or\cyrshch\or\cyrerev\or\cyryu\or
                \cyrya\else\@ctrerr\fi}
    \fi
    \makeatother
\fi


%%http://www.linux.org.ru/forum/general/6993203#comment-6994589 (используется totcount)
\makeatletter
\def\formtotal#1#2#3#4#5{%
    \newcount\@c
    \@c\totvalue{#1}\relax
    \newcount\@last
    \newcount\@pnul
    \@last\@c\relax
    \divide\@last 10
    \@pnul\@last\relax
    \divide\@pnul 10
    \multiply\@pnul-10
    \advance\@pnul\@last
    \multiply\@last-10
    \advance\@last\@c
    #2%
    \ifnum\@pnul=1#5\else%
        \ifcase\@last#5\or#3\or#4\or#4\or#4\else#5\fi
    \fi
}
\makeatother

\newcommand{\formbytotal}[5]{\total{#1}~\formtotal{#1}{#2}{#3}{#4}{#5}}

%%% Команды рецензирования %%%
\ifboolexpr{ (test {\ifnumequal{\value{draft}}{1}}) or (test {\ifnumequal{\value{showmarkup}}{1}})}{
    \newrobustcmd{\todo}[1]{\textcolor{red}{#1}}
    \newrobustcmd{\note}[2][]{\ifstrempty{#1}{#2}{\textcolor{#1}{#2}}}
    \newenvironment{commentbox}[1][]%
    {\ifstrempty{#1}{}{\color{#1}}}%
    {}
}{
    \newrobustcmd{\todo}[1]{}
    \newrobustcmd{\note}[2][]{}
    \excludecomment{commentbox}
}
