%%% Проверка используемого TeX-движка %%%

% определяем новый условный оператор (http://tex.stackexchange.com/a/47579)
\newif\ifxetexorluatex
\ifxetex
    \xetexorluatextrue
\else
    \ifluatex
        \xetexorluatextrue
    \else
        \xetexorluatexfalse
    \fi
\fi

% Условие, проверяющее, что документ --- автореферат
\newif\ifsynopsis

% Для продвинутой проверки разных условий
\usepackage{etoolbox}[2015/08/02]
\providebool{presentation}

% Позволяет убирать блоки текста (добавляет окружение comment и команду \excludecomment)
\usepackage{comment}    

%%% Поля и разметка страницы %%%

% Для включения альбомных страниц
\usepackage{pdflscape}

% Для последующего задания полей
\usepackage{geometry}

%%% Математические пакеты %%%
% Математические дополнения от AMS
\usepackage{amsthm,amsmath,amscd}

% Математические дополнения от AMS
\usepackage{amsfonts,amssymb}

% Добавляет окружение multlined
\usepackage{mathtools}

% Красивые дроби
\usepackage{xfrac}
\usepackage[
    locale = DE,
    list-separator       = {;\,},
    list-final-separator = {;\,},
    list-pair-separator  = {;\,},
    list-units           = single,
    range-units          = single,
    range-phrase={\text{\ensuremath{-}}},
    % красивые дроби могут не соответствовать ГОСТ
    % quotient-mode        = fraction,
    fraction-function    = \sfrac,
    separate-uncertainty,
    % Размерности SI
    ]{siunitx}
\sisetup{inter-unit-product = \ensuremath{{}\cdot{}}}

% Кириллица в нумерации subequations
% Для правильной работы требуется выполнение сразу после загрузки пакетов
\patchcmd{\subequations}{\def\theequation{\theparentequation\alph{equation}}}
{\def\theequation{\theparentequation\asbuk{equation}}}
{\typeout{subequations patched}}{\typeout{subequations not patched}}

%%%% Установки для размера шрифта 14 pt %%%%
%% Формирование переменных и констант для сравнения (один раз для всех подключаемых файлов)%%
%% должно располагаться до вызова пакета fontspec или polyglossia, потому что они сбивают его работу
\newlength{\curtextsize}
\newlength{\bigtextsize}
\setlength{\bigtextsize}{13.9pt}

\makeatletter

% неплохо для отслеживания, но вызывает стопорение процесса, если документ компилируется без команды  -interaction=nonstopmode
%\show\f@size

\setlength{\curtextsize}{\f@size pt}
\makeatother

%%% Кодировки и шрифты %%%
\ifxetexorluatex
    \ifpresentation
        % quick fix for polyglossia 1.50
        \providecommand*\autodot{}
    \fi
    % https://tex.stackexchange.com/a/26295/104425
    \PassOptionsToPackage{no-math}{fontspec}

    % Поддержка многоязычности
    % (fontspec подгружается автоматически)
    \usepackage{polyglossia}[2014/05/21]
\else
   %%% Решение проблемы копирования текста в буфер кракозябрами
    \ifnumequal{\value{usealtfont}}{0}{}{
        \input glyphtounicode.tex
        %from pdfx package
        \input glyphtounicode-cmr.tex
        \pdfgentounicode=1
    }
    % Улучшенный поиск русских слов в полученном pdf-файле
    \usepackage{cmap}
    \ifnumequal{\value{usealtfont}}{2}{}{
        % Если стоит до fontenc, то переносы не впишутся в выделяемый текст при копировании его в буфер обмена
        \defaulthyphenchar=127
    }

    % Поддержка русских букв
    \usepackage{textcomp}
    \usepackage[T1,T2A]{fontenc}                    

    % Используется pscyr, при наличии
    \ifnumequal{\value{usealtfont}}{1}{
        % Подключение pscyr
        \IfFileExists{pscyr.sty}{\usepackage{pscyr}}{}
    }{}

    % Кодировка utf8
    \usepackage[utf8]{inputenc}[2014/04/30]   
    
    % Языки: русский, английский
    \usepackage[english, russian]{babel}[2014/03/24]

    % babel 3.40 fix
    \makeatletter\AtBeginDocument{\let\@elt\relax}\makeatother
    \ifnumequal{\value{usealtfont}}{2}{
        % http://dxdy.ru/post1238763.html#p1238763
        % Подключение русифицированных шрифтов XCharter
        \usepackage[scaled=0.914]{XCharter}[2017/12/19]
        \usepackage[charter, vvarbb, scaled=1.048]{newtxmath}[2017/12/14]
        \ifpresentation
        \else
            \setDisplayskipStretch{-0.078}
        \fi
    }{}
\fi

%%% Оформление абзацев %%%
\ifpresentation
\else
    % Красная строка после заголовков типа chapter
    \indentafterchapter
    \usepackage{indentfirst}
\fi

%%% Цвета %%%
\ifpresentation
\else
    % Совместимо с tikz
    \usepackage[dvipsnames, table, hyperref]{xcolor}
\fi

%%% Таблицы %%%

% Длинные таблицы
\usepackage{longtable,ltcaption}

% Улучшенное форматирование таблиц
\usepackage{multirow,makecell}

% таблицы с автоматически подбирающейся шириной столбцов (tabu обязательно до hyperref вызывать)
\usepackage{tabu, tabulary}

% автоматический подгон ширины подписи таблицы
\usepackage{threeparttable}

%%% Общее форматирование

% Поддержка переносоустойчивых подчёркиваний и зачёркиваний
\usepackage{soulutf8}

% Запятая в десятичных дробях
\usepackage{icomma}

%%% Оптимизация расстановки переносов и длины последней строки абзаца

% проверка установленности пакета impnattypo
\IfFileExists{impnattypo.sty}{
    \ifluatex
        % Черновик
        \ifnumequal{\value{draft}}{1}
        {
            \usepackage[hyphenation, lastparline, nosingleletter, homeoarchy, rivers, draft]{impnattypo}
        }{
            % Чистовик
            \usepackage[hyphenation, lastparline, nosingleletter]{impnattypo}
        }
    \else
        \usepackage[hyphenation, lastparline]{impnattypo}
    \fi
}{}

%% Векторная графика

% Продвинутый пакет векторной графики
\usepackage{tikz}

% Для примера tikz рисунка
\usetikzlibrary{chains}

% Для примера tikz рисунка
\usetikzlibrary{shapes.geometric}

% Для примера tikz рисунка
\usetikzlibrary{shapes.symbols}

% Для примера tikz рисунка
\usetikzlibrary{arrows}

%%% Гиперссылки %%%
\ifxetexorluatex
    \let\CYRDZE\relax
\fi
\usepackage{hyperref}[2012/11/06]

%%% Изображения %%%
% Подключаем пакет работы с графикой
\usepackage{graphicx}[2014/04/25]

% Подписи рисунков и таблиц
\usepackage{caption}

% Подписи подрисунков и подтаблиц
\usepackage{subcaption}

% Добавление внешних pdf файлов
\usepackage{pdfpages}

%%% Счётчики %%%
\usepackage{aliascnt}

% Счётчик рисунков и таблиц
\usepackage[figure,table]{totalcount}

% Пакет создания счётчиков на основе последнего номера подсчитываемого элемента (может требовать дважды компилировать документ)
\usepackage{totcount}

% Счётчик страниц, совместимый с hyperref (ссылается на номер последней страницы)
% Желательно ставить последним пакетом в преамбуле
\usepackage{totpages}

%%% Продвинутое управление групповыми ссылками (пока только формулами) %%%
\ifpresentation
\else
    % cleveref имеет сложности со считыванием языка из babel.
    % Такое решение русификации вывода выбрано вместо определения в documentclass из опасности что-то лишнее передать во все остальные пакеты, включая библиографию.
    \usepackage[russian]{cleveref}

    % Добавление возможности использования пробелов в \labelcref
    % https://tex.stackexchange.com/a/340502/104425
    \usepackage{kvsetkeys}
    \makeatletter
    \let\org@@cref\@cref
    \renewcommand*{\@cref}[2]{%
        \edef\process@me{%
            \noexpand\org@@cref{#1}{\zap@space#2 \@empty}%
        }\process@me
    }
    \makeatother
\fi

% для \FloatBarrier
\usepackage{placeins}

% Черновик
\ifnumequal{\value{draft}}{1}{
    \usepackage[firstpage]{draftwatermark}
    \SetWatermarkText{DRAFT}
    \SetWatermarkFontSize{14pt}
    \SetWatermarkScale{15}
    \SetWatermarkAngle{45}
}{}

%%% Цитата, не приводимая в автореферате:

% возможно, актуальна только для biblatex
%\newcommand{\citeinsynopsis}[1]{\ifsynopsis\else ~\cite{#1} \fi}

% если текущий процесс запущен библиотекой tikz-external, то прекомпиляция должна быть включена
\ifdefined\tikzexternalrealjob
    \setcounter{imgprecompile}{1}
\fi

% Только если у нас включена предкомпиляция
\ifnumequal{\value{imgprecompile}}{1}{
    % подключение возможности предкомпиляции
    \usetikzlibrary{external}
    % activate! % здесь можно указать отдельную папку для скомпилированных файлов
    \tikzexternalize[prefix=images/cache/,optimize command away=\includepdf]
    \ifxetex
        \tikzset{external/up to date check={diff}}
    \fi
}{}
