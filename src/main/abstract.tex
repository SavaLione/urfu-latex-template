\chapter*{\centering РЕФЕРАТ}
\addcontentsline{toc}{chapter}{РЕФЕРАТ}

Развитие науки и техническое совершенствование выносит технологии \fixme{...} на абсолютно новый уровень.
Фантастическая скорость, с которой \fixme{...}, заставляет лишь удивляться непостижимыми возможностями,
которые доступны на данный момент и которые будут доступны в скором времени.

Тема работы -- \fixme{...}

Цель работы -- \fixme{...}

План проведения учебной педагогической практики состоит из трёх этапов:
\begin{enumeratePaper}
    \item Организационный этап.
    \item Основной этап.
    \item Заключительный этап.
\end{enumeratePaper}
Организационный этап:
\begin{itemizePaper}
    \item \fixme{...};
    \item \fixme{...};
\end{itemizePaper}

Объект исследования -- \fixme{...}.

Методы исследования: \fixme{...}.

Результаты работы:
\fixme{...}

Ключевые слова: \fixme{...}.

Объём и структура работы.
Отчёт состоит из~титульного листа,
реферата,
оглавления,
перечня условных обозначений,
введения,
\formbytotal{totalchapter}{глав}{ы}{}{},
заключения
и
библиографического списка.
Полный объём работы составляет
\formbytotal{TotPages}{страниц}{у}{ы}{}, включая
\formbytotal{totalcount@figure}{рисун}{ок}{ка}{ков}.
Список литературы содержит
\formbytotal{citenum}{наименован}{ие}{ия}{ий}.
