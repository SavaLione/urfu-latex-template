\chapter*{\centering Словарь терминов}
\addcontentsline{toc}{chapter}{Словарь терминов}


\textbf{3-­clause BSD} (3-­clause Berkeley Software Distribution license) - программная лицензия университета Беркли состоящая из трёх пунктов.

\textbf{API} (Application Programming Interface) - описание способов взаимодействия компьютерных программ между собой.

\textbf{Arch Linux} - дистрибутив Linux.

\textbf{Debian} - дистрибутив Linux.

\textbf{git} - распределённая система управления версиями.

\textbf{Linux} - семейство Unix-подобных операционных систем.

\textbf{Mac OS X} - семейство коммерческих операционных систем.

\textbf{REST} (Representation State Transfer) - архитектура, описывающая взаимодействие компонентов распределённого приложения в сети.

\textbf{TCP/IP} - сетевая модель передачи данных.

\textbf{Ubuntu} - дистрибутив Linux, основанный на Debian GNU/Linux.

\textbf{Unix} - семейство многопользовательских, многозадачных и переносимых операционных систем.

\textbf{Windows} - семейство коммерческих операционных систем.

