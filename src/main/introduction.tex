\chapter*{\centering ВВЕДЕНИЕ}
\addcontentsline{toc}{chapter}{ВВЕДЕНИЕ}    % Добавляем его в оглавление

Развитие науки и техническое совершенствование выносит технологии \fixme{...} на абсолютно новый уровень.
Фантастическая скорость, с которой \fixme{...}, заставляет лишь удивляться непостижимыми возможностями,
которые доступны на данный момент и которые будут доступны в скором времени.

Предполагаемая потребность в данной работе обусловлена тем, что на текущий момент
\fixme{...}

Разработанные \fixme{...}

Актуальность данной работы обусловлена \fixme{...}

Целью данной работы является
\fixme{...}

Основная гипотеза данной работы заключается в мнении, что
\fixme{...}
Стоит отметить и пользу для \fixme{...}, которая будет заключаться в:
\begin{itemizePaper}
    \item \fixme{...};
    \item \fixme{...}.
\end{itemizePaper}

Для достижения поставленной цели необходимо было
выполнить \fixme{...} и
решить следующие задачи:
\begin{itemizePaper}
    \item \fixme{...};
    \item \fixme{...}.
\end{itemizePaper}

В качестве теоретической и информационно-аналитической базы были использованы научные
труды авторов по заявленной тематике,
публикуемые в научных журналах,
одобренных высшей аттестационной комиссией или публикуемые в библиографических и реферативных базах данных Scopus и Web of Science,
статьи,
входящие в сборник трудов конференций, информационные материалы,
размещённые на сайтах сети Интернет.
